\documentclass{article}
\usepackage[utf8]{inputenc}
\usepackage[a4paper]{geometry}
\geometry{hscale=1.02,vscale=1,centering}
\usepackage{multirow}
\usepackage[table]{xcolor}

\begin{document}

\begin{tabular}{|c||c||c||c||c|} \hline
\rowcolor{green}
\parbox[][1.5cm][c]{1.7cm}{Lundi} & \parbox[][1.5cm][c]{1.7cm}{Mardi}
& \parbox[][1.5cm][c]{1.7cm}{Mercredi} & \parbox[][1.5cm][c]{1.7cm}{Jeudi} & \parbox[][1.5cm][c]{1.7cm}{Vendredi}\\ \hline

\parbox[][2.5cm][c]{3.7cm}{Présentation croisée : chacun dit quelques mots de description sur son binôme
   On met de côté pour réutiliser mercredi.} & \parbox[][1.5cm][c]{3.7cm}
{Temps de discussion sur la veille. 15'. } & \parbox[][1.5cm][c]{3.7cm}{ Temps de discussion sur la veille. 15'.} &  \parbox[][2.5cm][c]{3.7cm}{Temps de discussion sur la veille. 15'}
&\multirow{7}{*}{
     \parbox[][25cm][c]{2.5cm}{Théâtre filmé:\\
     Choisir parmi les situations délaissées dans les théâtres forum des séances 3 et 4. La faire jouer. Filmer.}
    }
\\ 

\cline{1-4}
    \parbox[][1.5cm][c]{3.7cm}{ Où nous rangeons-nous? 15'} &
    \parbox[][1.5cm][c]{3.7cm}{Dessine-moi un sapin. 15'} &
    \parbox[][1.5cm][c]{3.7cm}{Commentaire de la fiche de présentation du binôme. } &
    \parbox[][4.1cm][c]{3.7cm}{Rôles attribués aux femmes / hommes. 40'
\begin{itemize}
    \item les rôles.
    \item clichés de genre exact pour moi ?
    \item tu ne rentres pas dans les cases.
\end{itemize}} &\\ 
    
\cline{1-4}
\parbox[][4cm][c]{3.7cm}{Comparaison d'ADN (pour homme : salive)
   \begin{itemize}
       \item banane-homme
       \item blanc-noir 
       \item homme-femme
       \item blond-brun
   \end{itemize}}
   
   & \parbox[][4cm][c]{3.7cm}{Trier des objets: 30'
   \begin{itemize}
       \item libre.
       \item selon des critères.
   \end{itemize}} 
   
   & \parbox[][5.5cm][c]{3.7cm}{Stéréotypes et préjugés:
   \begin{itemize}
        \item Qui fait quoi ? 15' (ce qu'on pense des gens en voyant leur look)
        \item Je ne suis pas... 15' (Le préjugé dont vous avez souffert) -- sous forme de smswall 
   \end{itemize}} }
   
   &\parbox[][4cm][c]{3.7cm}{Stéréotype du genre : 1h15. Quels impacts sur la personnalité.
    \begin{itemize}
        \item Les métiers ont un sens.
        \item Performance.
    \end{itemize}}
    &
    \\

\cline{1-4}
 \cellcolor{yellow}
   \parbox[][4cm][c]{3.7cm}{BILAN INTERMEDIAIRE : Beaucoup de choses nous rapprochent des autres êtres vivants. Nous sommes semblables biologiquement quelque soit le sexe, la race ou l'apparence.}
   
   &\cellcolor{yellow}
   \parbox[][4cm][c]{3.7cm}{ BILAN INTERMEDIAIRE : pour réfléchir vite, on fait des catégorisations. Mais tout le monde ne fait pas les mêmes.}
   
   &\cellcolor{yellow} 
   \parbox[][4cm][c]{3.7cm}{BILAN INTERMEDIAIRE : Tout le monde a des préjugés et se trouve l'objet de préjugés.}
   
   &\cellcolor{yellow}
   \parbox[][4cm][c]{3.7cm}{BILAN INTERMEDIRAIRE : La discrimination selon le sexe est la plus répandue et la plus courante. Nous la vivons tous.}
   &
   \\
 
\cline{1-4}
   \parbox[][4cm][c]{3.7cm}{Qu'est-ce qui nous distingue des primates? 30'}
   
   &\parbox[][4cm][c]{3.7cm}{Les phrases stéréotypes : généraliser, c'est encore un stéréotype. 15'}
   
   & \parbox[][4cm][c]{3.7cm}{Taxi-Taxi. 30'}
   
   &\multirow{2}{*}{
    \parbox[][4cm][c]{3.7cm}{Thêatre forum sur les inégalités femmes/hommes. 1h\\ 
Conclusion : Ca vous a plu ? Demain on remet ça et on film !}}
   &
   \\
\cline{1-3}

   \parbox[][4cm][c]{3.7cm}{5 sens et perception. D'un individu à l'autre, les perceptions diffèrent. 30'}
   & \parbox[][4cm][c]{3.7cm}{Reconnaissance d'un visage. 45'
   Post-it sur des photos. 2 informations qu'on peut en tirer: 
   \begin{itemize}
       \item description
       \item jugement
   \end{itemize}}
   
   & \parbox[][4cm][c]{3.7cm}{Déjouons la discrimination. 45' \\ Sénette par groupe de 3-4 personnes. On joue une situation de discrimination. (+débat après)}
    
    &  
    
    &
   \\
  
\cline{1-4}
\cellcolor{magenta}
   \parbox[][4cm][c]{3.7cm}{OBJECTIF: L'homme est un élément de la nature, un animal parmi d'autres. Il se distingue d'eux, même des plus proches. Chacun de nous percoit le monde différemment.}
   
   & \cellcolor{magenta}
   \parbox[][4cm][c]{3.7cm}{OBJECTIF : Faire des catégorisations, c'est normal, de même qu'avoir des stéréotypes ancrés en nous. Il ne faut pas confondre description et jugement.}
   
   & \cellcolor{magenta}
   \parbox[][4cm][c]{3.7cm}{OBJECTIF : Les stéréotypes peuvent mener à la discrimination. Chacun peut être à la fois victime et coupable de la discrimination. Mais ça peut se corriger.}

  &\cellcolor{magenta}
  \parbox[][4cm][c]{3.7cm}{OBJECTIF : Parmi les discriminations, celles qui touchent les femmes sont les plus répandues. Basées elles aussi sur des stéréotypes.}
  
  &
   \\
   
\hline
    






   

   

  

\end{tabular}

\end{document}
